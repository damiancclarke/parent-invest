\documentclass{article}[12pt,subeqn]

%================================================================================
%PACKAGES
%================================================================================
\usepackage{amsmath}
%\usepackage{appendix}
\usepackage{array}
\usepackage{booktabs}
%\usepackage{cleveref}
\usepackage[usenames, dvipsnames]{color} 
\usepackage{framed}
\usepackage[capposition=top]{floatrow}
\usepackage[pdftex]{graphicx}
\usepackage[update,prepend]{epstopdf}
\usepackage{hyperref}
\usepackage[utf8]{inputenc}
%\usepackage{lineno}
\usepackage{lscape}
\usepackage{natbib}
\usepackage{rotating,capt-of}
\usepackage{setspace}
\usepackage{siunitx}
\usepackage{subfloat}
\usepackage{caption}
\usepackage{subcaption}

\bibliographystyle{abbrvnat}
\bibpunct{(}{)}{;}{a}{,}{,}


%================================================================================
%TITLE
%================================================================================


%================================================================================
%MARGINS
%================================================================================
\setlength\topmargin{-0.375in}
\setlength\textheight{8.8in}
\setlength\textwidth{5.8in}
\setlength\oddsidemargin{0.4in}
\setlength\evensidemargin{-0.5in}
\setlength\parindent{0.25in}
\setlength\parskip{0.25in}


%================================================================================
%NEW COMMANDS
%================================================================================
\newcommand{\Lagr}{\mathcal{L}}
\newcommand{\vect}[1]{\mathbf{#1}}
\newcolumntype{P}[1]{>{\raggedright}p{#1\linewidth}}

\hypersetup{                                                                                                    
    colorlinks=true,   
    linkcolor=BlueViolet,
    citecolor=BlueViolet,
    filecolor=BlueViolet,
    urlcolor=BlueViolet
}  



\begin{document}
\begin{spacing}{1.4}
\section{Arsenic Exposure in the North of Chile: History}
Arsenic is present naturally in the earth's crust, and is principally
produced by volcanic activity.  Arsenic is widely found in natural
ground water, although concentrations are generally low: typically
below \SI{10}{\micro\gram}/L, or 10 parts per billion \citep{WHO2001}.
However, in some circumstances, particularly in volcanic regions,
geological water sources contain much higher concentrations of the
metalloid.\footnote{Elemental arsenic (As) is not soluble in water.
Soluble arsenic in drinking water is typicall found as arsenate salts,
a form of anionic arsenic.} 

\subsection{Arsenic in Drinking Water in Antofagasta}
Such is the case for the north of Chile, a zone spanned by the Atacama
desert, where all drinking water consumed is drawn from ground
water. Geographically, Chile is split into 15 regions (similar to   
States in the USA), numbered from I to XV.  The five northernmost
regions (XV, I, II, III, and IV), house 2,244,666 people, or 12.6\% of
Chile's population of 17,819,054 \citep{INE2014}.  The Atacama desert,
which contains these five regions, is the driest desert in the world,
receiving approximately 15 mm of rainfall per year.  Historically,
extremeley low rainfall and a growing population has meant that a
number of important changes in water sources occurred.

Drawing water from alternative sources has meant that the
concentration of arsenic in drinking water has fluctuated
substantially historically.  This is particularly the case for
Antofagasta (the fifth-largest city in Chile) and its surrounds.
From its foundation in 1860, Antofagasta obtained drinking water by
distilling sea water.  With the growth of the population, the
water sources were expanded to include the Siloli river, whose arsenic
concentration varies from 90 to \SI{100}{\micro\gram}/L, and this
source was used until the late 1950s \citep{FerreccioSancha2006}. 


With further population growth, the city attempted to secure a more
abundant source of water by constructing a series of pipes to
transport water 300 km from the Toconce river.  The first water
arrived from the Toconce river in August of 1958
\citep{MainoRecabarren2012,FerreccioSancha2006}. However,
concentrations of Arsenic in the Toconce river were extremely high:
ranging from 800 to \SI{900}{\micro\gram}/L.  Hence, between 1958 and
1959, the concentrations consumed by individuals in the town jumped
nearly 10 fold (to 17 times the WHO recommendation), with the majority
of water consumed by inhabitants drawn from the new source
\citep{Fraser2012}. 

Extremely high concentrations continued to be consumed from the end of
1958 until 1971.  In 1971, due to growing realisations of the harm
caused by arsenic in drinking water, along with campaigns in regional
newspapers, the first arsenic removal plant was put in operation in
Arsenic.  This plant reduced the concentrations of arsenic in drinking
water to approximately \SI{110}{\micro\gram}/L, lower than the newly
implemented law requiring concentrations below \SI{120}{\micro\gram}/L
in drinking water.  The solid line in figure \ref{ASfig:arsenic} plots
the concentration of arsenic in drinking water in Antofagasta and its
surrounds between 1930 and 2002, with the important spike
corresponding to the use of Toconce river water between 1958 and 1971.

\begin{figure}[htpb]
\includegraphics{./arsenicConcentrations.eps}
\caption{Historical Concentrations of Arsenic in Tap Water (\SI{}{\micro\gram}/L)}
\label{ASfig:arsenic}
\end{figure}

\subsection{Arsenic Concentrations in Other Cities}
With population growth in other areas of the northern regions of
Chile, an additional arsenic removal plant was constructed in 1978
\citep{MainoRecabarren2012}. This plant, treating water for the cities
of Calama and Tocopilla, reduced arsenic concentration in these cities
(see dashed line in figure \ref{ASfig:arsenic}).  A third and final
plant was constructed in Antofagasta in 1989 to increase the capacity
for arsenic removal in the city.

Technological improvements in arsenic removal technologies along with
an increasing awareness of the damage caused by arsenic consumption,
resulted in the modification of the Chilean Law for Drinking Water in
1984.  The old requirement that arsenic concentrations must be no
greater than 120 ppb was reduced to a maximum level of 50
ppb. Finally, in 2005 the law was modified again, to comply with the
international WHO requirement that arsenic concetrations do not exceed
10 ppb \citep{WHO2001}.  Currently, with the exception of some very
remote and small communities (ie San Pedro) where alternative potable water
arrangements are in place, all tap water in Chile complies the the WHO
and Chilean normative, containing less than \SI{10}{\micro\gram}/L.




\newpage
\begin{figure}
\centering
\begin{subfigure}{.5\textwidth}
  \centering
  \includegraphics[scale=0.4,width=.4\linewidth]{ArsenicMap.pdf}
  \caption{Arsenic Average Concentrations: 1930-2002}
  \label{fig:sub1}
\end{subfigure}%
\begin{subfigure}{.5\textwidth}
  \centering
  \includegraphics[scale=0.4,width=.4\linewidth]{ArsenicChange.pdf}
  \caption{Change in Arsenic Concentrations (Max-Min)}
  \label{fig:sub2}
\end{subfigure}
\caption{Arsenic in Chile}
\label{fig:test}
\end{figure}


\clearpage

\newpage
\bibliography{./refs}

\end{spacing}
\end{document}
