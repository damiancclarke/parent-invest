\documentclass{article}[12pt,subeqn]

%================================================================================
%PACKAGES
%================================================================================
\usepackage{amsmath}
%\usepackage{appendix}
\usepackage{array}
\usepackage{booktabs}
%\usepackage{cleveref}
\usepackage[usenames, dvipsnames]{color} 
\usepackage{framed}
\usepackage[capposition=top]{floatrow}
\usepackage[pdftex]{graphicx}
\usepackage[update,prepend]{epstopdf}
\usepackage{hyperref}
\usepackage[utf8]{inputenc}
%\usepackage{lineno}
\usepackage{lscape}
\usepackage{natbib}
\usepackage{rotating,capt-of}
\usepackage{setspace}
\usepackage{siunitx}
\usepackage{subfloat}

\bibliographystyle{abbrvnat}
\bibpunct{(}{)}{;}{a}{,}{,}


%================================================================================
%TITLE
%================================================================================


%================================================================================
%MARGINS
%================================================================================
\setlength\topmargin{-0.375in}
\setlength\textheight{8.8in}
\setlength\textwidth{5.8in}
\setlength\oddsidemargin{0.4in}
\setlength\evensidemargin{-0.5in}
\setlength\parindent{0.25in}
\setlength\parskip{0.25in}


%================================================================================
%NEW COMMANDS
%================================================================================
\newcommand{\Lagr}{\mathcal{L}}
\newcommand{\vect}[1]{\mathbf{#1}}
\newcolumntype{P}[1]{>{\raggedright}p{#1\linewidth}}

\hypersetup{                                                                                                    
    colorlinks=true,   
    linkcolor=BlueViolet,
    citecolor=BlueViolet,
    filecolor=BlueViolet,
    urlcolor=BlueViolet
}  



\begin{document}
\begin{spacing}{1.4}
\section{Arsenic Exposure in the North of Chile: History}
Arsenic is present naturally in the earth's crust, and is principally
produced naturally due to volcanic activity.  Arsenic is widely found
in natural ground water, although concentrations are generally low:
typically below \SI{10}{\micro\gram}/L, or 10
parts per billion \citep{WHO2001}.  However, in some circumstances,
particulalry in volcanic regions, geological water sources contain
much higher concentrations of the metalloid.\footnote{Elemental arsenic
(As) is not soluble in water.  Soluble arsenic in drinking water is
typicall found as arsenate salts, a form of anionic arsenic.}

Such is the case for the north of Chile, where 




\newpage
\bibliography{./refs}

\end{spacing}
\end{document}
